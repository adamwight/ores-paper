Intelligent algorithms have a long history of making curation work in peer production tractable.  From counter-vandalism to task routing, basic machine prediction allows open knowledge projects like Wikipedia to scale to the largest encyclopedia in the world.  However, the ideologies and values of the community were captured in the development of these algorithms and the processes they support.  Wikipedia's challenges and the community's values have changed in the last decade, but its algorithmic support systems have remained largely stagnant.  The conversation about what quality control should be and what place algorithms have remains restricted to a few expert engineers.  In this paper, we describe ORES: an algorithmic service designed to open up socio-technical conversations in Wikipedia to a broader set of participants.  In this paper, we argue the theoretical mechanisms of social change ORES enables and we describe the phenomena around ORES from the 3 years since ORES' deployment.
